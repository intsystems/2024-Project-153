\documentclass{article}
\usepackage{arxiv}

\usepackage[T2A]{fontenc}			
\usepackage[utf8]{inputenc}			
\usepackage[english,russian]{babel}	
\usepackage{amsmath,amsfonts,amssymb,amsthm,mathrsfs,mathtools} 
\usepackage{url}
\usepackage{booktabs}
\usepackage{nicefrac}
\usepackage{microtype}
\usepackage{lipsum}
\usepackage{graphicx}
\usepackage{natbib}
\usepackage{doi}
\usepackage{comment}



\title{Погружение временных рядов с высокой волатильностью в метрическое пространство}

\author{ Эйнуллаев Алтай \\
	Кафедра интеллектуальных систем\\
	Московский физико-технический институт\\
	Долгопрудный \\
	\texttt{einullaev.ae@phystech.edu} \\
	%% examples of more authors
	\And
	Яковлев Константин \\
	Кафедра интеллектуальных систем\\
	Московский физико-технический институт\\
	Долгопрудный \\
	\texttt{iakovlev.kd@phystech.edu} \\
	%% \AND
	%% Coauthor \\
	%% Affiliation \\
	%% Address \\
	%% \texttt{email} \\
	%% \And
	%% Coauthor \\
	%% Affiliation \\
	%% Address \\
	%% \texttt{email} \\
	%% \And
	%% Coauthor \\
	%% Affiliation \\
	%% Address \\
	%% \texttt{email} \\
}
\date{}

\renewcommand{\shorttitle}{\textit{arXiv} Template}

%%% Add PDF metadata to help others organize their library
%%% Once the PDF is generated, you can check the metadata with
%%% $ pdfinfo template.pdf
\hypersetup{
pdftitle={A template for the arxiv style},
pdfsubject={q-bio.NC, q-bio.QM},
pdfauthor={David S.~Hippocampus, Elias D.~Striatum},
pdfkeywords={First keyword, Second keyword, More},
}

\begin{document}
\maketitle

\begin{abstract}
	Рассматривается задача прогнозирования финансовых временных рядов. Основными особенностями таких временных рядов являются высокая волатильность и высокая попарная ковариация. Классическим подходом к решению задачи является выполнение прогноза в исходном пространстве. Новый метод заключается в переходе в пространство попарных расстояний между временными рядами, осуществлении прогноза в нем и переходе обратно в исходное пространство. Для его реализации необходимо ввести функцию расстояния между временными рядами, которая должна удовлетворять определенным свойствам. В данной статье изучаются  эти свойства и проводятся сравнения различных метрик на основе численных экспериментов.

\end{abstract}


\keywords{Временные ряды \and Метрика \and Ковариация}

\section{Introduction}

В текущей статье исследуется задача погружения временных рядов в метрическое пространство. Набору временных рядов ставится в соответствие матрица попарных расстояний и появляется возможность перейти от прогнозирования набора временных рядов к прогнозированию матрицы попарных расстояний. При этом выбор функции расстояния осуществляется так, чтобы по полученной матрице расстояний можно было восстановить прогноз для набора временных рядов.

В статистике, обработке сигналов и многих других областях под временным рядом понимаются последовательно измеренные через некоторые (зачастую равные) промежутки времени данные. \cite{shumway2000time} Прогнозирование временных рядов заключается в построении модели для предсказания будущих событий основываясь на известных событиях прошлого, предсказания будущих данных до того как они будут измерены.

Одними из хорошо известных, классических методов прогнозирования временных рядов являются экспоненциальное сглаживание (англ. Exponential Smoothing) \cite{ES}, LSTM (англ. Long Short-Term Memory) \cite{LSTM}, ARIMA  (англ. autoregressive integrated moving average) \cite{ARIMA}. Главным отличием исследуемого метода от вышеперечисленных является то, что временные ряды прогнозируются при помощи прогнозирования матрицы попарных расстояний.

В качестве простейшей метрики рассматривается ковариация между временными рядами. \cite{Boyd} Таким образом, для набора временных рядов получаем матрицу ковариации. Стоит заметить, что матрица ковариации (матрица попарных расстояний) вычисляется в каждый момент времени. Альтернативные варианты метрики выбираются из класса ядер \cite{shawe2004kernel}.

Численные эксперименты проводятся на трех видах данных: синтетические, сигналы коры головного мозга, финансовые временные ряды. Эксперимент состоит из выполнения прогноза временного ряда при помощи прогнозирования матрицы попарных расстояний. В качестве прогностической модели рассматривается ARIMA и LSTM. По результатам экспериментов проводится анализ точности прогноза и его устойчивости в зависимости от выбранной метрики и вида данных. Цель эксперимента состоит в оптимальном выборе функции попарных расстояний для выполнения прогноза.

\section{Problem statement}

Пусть \text{X} $ = \{\mathbf{x} = [x_1, \ldots, x_\text{m}]^T | x_i \in R\}$ --- множество временных рядов, заданных своей реализацией. Обозначим через  $\mathbf{Y} \in R^{n \times m}$ заданный набор из $n$ временных рядов:

\begin{equation}
    \mathbf{Y} = [\mathbf{x}^{(1)}, \ldots, \mathbf{x}^{(n)}]^T.
\end{equation}

 Через $\mathbf{Y}_t \in R^{n \times t}$ обозначим $t < m$ первых столбцов $\mathbf{Y}$: 
 
 \begin{equation}
    \mathbf{Y}_t = [\mathbf{x}_{1:t}^{(1)}, \ldots, \mathbf{x}_{1:t}^{(n)}]^T.
\end{equation}. 

Определим функцию расстояния между временными рядами: $d : \text{X} \times \text{X} \rightarrow R$, удовлетворяющую условиям Мерсера \cite{ghojogh2021reproducing}:

1. $d(\mathbf{x}^{(1)}, \mathbf{x}^{(2)}) = d(\mathbf{x}^{(2)}, \mathbf{x}^{(1)})$ $\forall \mathbf{x}^{(1)}, \mathbf{x}^{(2)} \in \text{X}$.

2. $\forall n \in N$, $\forall \mathbf{x}^{(1)}, \ldots, \mathbf{x}^{(n)} \in X$ матрица $\Sigma \in S_n$, составленная из попарных расстояний между элементами, является неотрицательно определенной.

Обозначим расстояние между временными рядами $\mathbf{x}^{(i)} = [x_1^{(i)}, \ldots, x_t^{(i)}], \mathbf{x}^{(j)} = [x^{(j)}_1, \ldots, x^{(j)}_t]$ следующим образом:

\begin{equation}
    d_t(\mathbf{x}^{(i)}, \mathbf{x}^{(j)}) = d_t(i, j)
\end{equation}

Таким образом, в каждый момент времени $t$ набору временных рядов $\mathbf{Y}_t$ поставлена в соответствие матрица попарных расстояний $\Sigma_t \in \mathcal{S}_n^+$ (симметричная, неотрицательно определенная матрица):

\begin{equation}
    \mathbf{Y}_t \Rightarrow \Sigma_t = \left(
\begin{array}{cccc}
d_t(\mathbf{x}^{(1)}, \mathbf{x}^{(1)}) & d_t(\mathbf{x}^{(1)}, \mathbf{x}^{(2)}) & \ldots & d_t(\mathbf{x}^{(1)}, \mathbf{x}^{(n)})\\
d_t(\mathbf{x}^{(2)}, \mathbf{x}^{(1)}) & d_t(\mathbf{x}^{(2)}, \mathbf{x}^{(2)}) & \ldots & d_t(\mathbf{x}^{(2)}, \mathbf{x}^{(n)})\\
\vdots & \vdots & \ddots & \vdots\\
d_t(\mathbf{x}^{(n)}, \mathbf{x}^{(1)}) & d_t(\mathbf{x}^{(n)}, \mathbf{x}^{(2)}) & \ldots & d_t(\mathbf{x}^{(n)}, \mathbf{x}^{(n)})\\
\end{array}
\right)
\end{equation}

Дадим описание двум используемым прогностическим моделям.

\subsection{ARIMA}

Пусть имеем реализацию матрицы попарных расстояний из \text{m} компонент: $\Sigma = [\Sigma_1, \Sigma_2, \ldots, \Sigma_\text{m}]$. Пусть $\Sigma_{\text{m} + 1}$ - прогнозируемая матрица. Авторегрессионная модель:

\begin{equation}
    \hat{\Sigma}_{\text{m} + 1} = \sum\limits_{i = 1}^m a_i\Sigma_{i} + \epsilon_{\text{m} + 1}
\end{equation}

где $a_1, \ldots, a_\text{m} \in R$ --- параметры модели (коэффициенты авторегрессии), $\epsilon_{\text{m} + 1} \in R^{\text{m} \times \text{m}}$ --- белый шум. Обозначим через $mathbf{a} \in R^{\text{m}}$ вектор составленный из параметров модели. Вектор оптимальных параметров модели $\hat{\mathbf{a}} \in R^{\text{m}}$ определяется с помощью решения задачи оптимизации:

\begin{equation}
    \hat{\mathbf{a}} = \arg\min\limits_{\mathbf{a} \in R^{m \times m}}\|\Sigma_{\text{m} + 1} - \hat{\Sigma}_{\text{m} + 1}\|_F^2.
\end{equation}

\subsection{LSTM}

\begin{comment}
    \subsection{Многомерная гусеница (MSSA)}

Пусть имеем реализацию матрицы попарных расстояний из \text{m} компонент: $\Sigma = [\Sigma_1, \Sigma_2, \ldots, \Sigma_\text{m}]$. Определим \textbf{L} - ширина окна, \text{K} $ = \textbf{m} - \text{L} + 1$. В силу того, что матрица симметричная, то для каждой компоненты матрицы попарных расстояний, где $i \leqslant j$, построим матрицу траекторий (матрица Ганкеля):

\begin{equation}
    \mathbf{H}_{ij} = \left(
\begin{array}{cccc}
d_1(\mathbf{x}^{(i)}, \mathbf{x}^{(j)}) & d_2(\mathbf{x}^{(i)}, \mathbf{x}^{(j)}) & \ldots & d_{\text{K}}(\mathbf{x}^{(i)}, \mathbf{x}^{(j)})\\
d_2(\mathbf{x}^{(i)}, \mathbf{x}^{(j)}) & d_3(\mathbf{x}^{(i)}, \mathbf{x}^{(j)}) & \ldots & d_{\text{K} + 1}(\mathbf{x}^{(i)}, \mathbf{x}^{(j)})\\
\vdots & \vdots & \ddots & \vdots\\
d_{\text{L}}(\mathbf{x}^{(i)}, \mathbf{x}^{(j)}) & d_{\text{L} + 1}(\mathbf{x}^{(i)}, \mathbf{x}^{(j)}) & \ldots & d_\text{m}(\mathbf{x}^{(i)}, \mathbf{x}^{(j)})\\
\end{array}
\right).
\end{equation}

Строим блочную матрицу траекторий:

\begin{equation}
    \mathbf{H} = [\mathbf{H}_{11} : \mathbf{H}_{12} : \mathbf{H}_{13} : \mathbf{H}_{1n} : \mathbf{H}_{22} : \ldots : \mathbf{H}_{nn}]
\end{equation}

Сингулярное разложение симметричной, положительно полуопределенной матрицы $\mathbf{H}\mathbf{H}^T = \mathbf{U}\Lambda\mathbf{U}^T$, где $\Lambda$ - диагональная матрица $\text{L} \times \text{L}$ собственных значений $\lambda_1 \geq \ldots \geq \lambda_{\text{L}}$, $\mathbf{U} = [\mathbf{u}_1, \ldots, \mathbf{u}_{\text{L}}]$ - ортогональная матрица собственных векторов матрицы $\mathbf{H}\mathbf{H}^T$. Пусть $d = \max\{i|\lambda_i > 0, i \in \{1, \ldots, \text{L}\}\}$. Тогда исходную матрицу траекторий можно представить в виде:

\begin{equation}
    \mathbf{H} = \sum\limits_{i = 1}^d\sqrt{\lambda_i}\mathbf{u}_i\mathbf{v}_i^T = \mathbf{H}^{(1)} + \ldots + \mathbf{H}^{(d)} = \sum\limits_{i = 1}^r \mathbf{H}^{(i)} + \sum\limits_{i = r + 1}^d \mathbf{H}^{(i)},
\end{equation}

 где $\mathbf{v}_i = \mathbf{H}^T\mathbf{u}_i/\sqrt{\lambda_i}$ $\mathbf{H}_r = \sum\limits_{i = 1}^{r} \mathbf{H}^{(i)}$ --- сигнальные компоненты, а $\mathbf{H} - \mathbf{H}_r$ --- 
шумовые. Полученную матрицу можно представить в виде:

\begin{equation}
    \mathbf{H}_r = [ \widetilde{\mathbf{H}}_{11} : \widetilde{\mathbf{H}}_{12} : \ldots : \widetilde{\mathbf{H}}_{nn}]
\end{equation}

где каждая из матриц $\widetilde{\mathbf{H}}_{ij} \in R^{\text{m} - \text{L} + 1 \times \text{L}}$. Восстановим исходные расстояния с помощью антидиагонального усреднения элементов соответствующих матриц:

\begin{equation}
    \widetilde{d}_t(i, j) = \dfrac{1}{2t - 1} \sum\limits_{k, l: k + l - 1 = t} \widetilde{h}^{(ij)}_{kl}, t \in \{1, \ldots, m\}
\end{equation}

где $\widetilde{h}^{(ij)}_{kl}$ --- элемент на пересечении $k$-ой строки и $l$-го столбца матрицы $\mathbf{H}_{ij}$. Обозначим через $\mathbf{u}^{\triangledown}_j \in R^{\text{L} - 1}$ первые $\text{L - 1}$ компоненты собственного вектора $\mathbf{u}_j$, а $\pi_j \in R$ --- последнюю компоненту вектора $\mathbf{u}_j$, ($j = 1, \ldots, r$). Определим $v^2 = \sum\limits_{j = 1}^r \pi_{j}^2$ и $\mathbf{p} \in R^{\text{L} - 1}$:

\begin{equation}
    \mathbf{p} = \dfrac{1}{1 - v^2} \sum\limits_{j = 1}^r\pi_j\mathbf{u}_j^{\triangledown}.
\end{equation}

При $v^2 < 1$ возможен прогроз:
\begin{equation}
    [\widetilde{d}_{\text{m} + 1}(1, 1), \widetilde{d}_{\text{m} + 1}(1, 2), \ldots, \widetilde{d}_{\text{m} + 1}(n, n)]^T = \mathbf{p}^T\mathbf{Z}
\end{equation}

где $\mathbf{Z} \in R^{(\text{L} - 1) \times (\frac{n(n + 1)}{2})}$, $\mathbf{Z} = [\mathbf{z}^{(11)}, \ldots, \mathbf{z}^{(nn)}]$, $\mathbf{z}^{(ij)} = [\widetilde{d}_{\text{m} - \text{L} + 2}(i, j), \ldots, \widetilde{d}_{\text{m}}(i, j)]^T$. Таким образом, получаем:

\begin{equation}
    \hat{\Sigma}_{\text{m} + 1} = [\widetilde{d}_{\text{m} + 1}(i, j)]_{i, j = 1}^{n}
\end{equation}

где $\widetilde{d}_{\text{m} + 1}(i, j) = \widetilde{d}_{\text{m} + 1}(j, i)$, при $i > j$.
\end{comment}


\subsection{Задача прогнозирования набора временных рядов}

В качестве критериев качества прогноза временных рядов используются 

\begin{equation}
    \text{MAPE} = \frac{1}{m}\sum\limits_{i = 1}^{m}\dfrac{|x^{(i)}_t - \hat{x}^{(i)}_t|}{|x^{(i)}_t|},
\end{equation}

\begin{equation}
    \text{MSE} = \frac{1}{m}\sum\limits_{i = 1}^{m}(x^{(i)}_t - \hat{x}^{(i)}_t)^2.
\end{equation}

Также выбираются $p < m$ временных рядов из набора так, чтобы отказ от прогнозирования соответствующих $m - p$ временных рядов существенно повышал качество \text{MAPE}, \text{MSE}. 

\subsection{Выбор функции попарных расстояний}

Пусть $\mathcal{F} = \{d^{(1)}, \ldots, d^{(s)}\}$ --- множество функций попарных расстояний, из которого нужно выбрать оптимальный вариант. Пусть $\hat{\Sigma}(d)$ --- спрогнозированная, а $\Sigma(d)$ --- точная матрица попарных расстояний при использовании $d \in \mathcal{F}$. Тогда оптимальный по точности выбор функции попарных расстояний:

\begin{equation}
    d_{\text{acc}} = \arg\min\limits_{d \in \mathcal{F}} \sigma^2(d) = \arg\min\limits_{d \in \mathcal{F}} \|\hat{\Sigma}(d) - \Sigma(d)\|_{\text{F}}^2.
\end{equation}

Пусть $\widetilde{\mathbf{x}}^{(i)} = \mathbf{x}^{(i)} + \varepsilon_i$, $i = 1, \ldots, n$ --- зашумленный исходный набор временных рядов, где $\varepsilon_i \in R^\text{N}$. Пусть $\widetilde{\sigma}^2(d)$ --- точность прогноза матрицы попарных расстояний зашумленного набора временных рядов при использовании $d \in \mathcal{F}$. Тогда наболее устойчивый выбор функции попарных расстояний:

\begin{equation}
    d_{\text{stable}} = \arg\min\limits_{d \in \text{F}}|\widetilde{\sigma}^2(d) - \sigma^2(d)|
\end{equation}

\section{Computational experiment}

Целью эксперимента является сравнение различных способов вычисления матрицы попарных расстояний, путем анализа ошибки прогноза. В качестве прогностических модели рассмотрим LSTM, ARIMA. Проведем эксперименты как на синтетических наборах временных рядов, так и на реальных. Точность прогноза будем оценивать с помощью MAPE (15) и MSE (16). 

\subsection{Прогнозирование матрицы попарных расстояний}

Рассмотрим три способа вычисления матрицы попарных расстояний между временными рядами. Считаем, что в каждый момент времени набор временных рядов задается матрицей $\textbf{X} \in R^{n \times L}$:


\begin{equation}
\textbf{X} = \left( \begin{array}{cccc}
x^{(1)}_{t - \text{L} + 1} & x^{(1)}_{t - \text{L} + 2} & \ldots & x^{(1)}_{t} \\
x^{(2)}_{t - \text{L} + 1} & x^{(2)}_{t - \text{L} + 2} & \ldots & x^{(2)}_{t} \\
\ldots & \ldots & \ldots & \ldots \\
x^{(n)}_{t - \text{L} + 1} & x^{(n)}_{t - \text{L} + 2} & \ldots & x^{(n)}_{t}.
\end{array}\right)
\end{equation}


$i$-й столбец матрицы $\textbf{X}$ обозначим через вектор $\textbf{y}^{i}$. А $i$-ю строку матрицы $\textbf{X}$ обозначим через вектор $\textbf{x}^{(i)}$.   В каждый момент времени поставим в соответствие данному набору матрицу $\Sigma_t \in R^{\text{n} \times \text{n}}$, элементы которой описывают взаимосвязь различных компонент набора временных рядов. Рассмотрим три способа построения данной матрицы.

\subsubsection{Способ № 1}


\begin{equation}
\Sigma_t^{1} = \dfrac{1}{L}\sum\limits_{i = 1}^{L} \textbf{y}^{i} \textbf{y}^{iT}
\end{equation}

\subsubsection{Способ № 2}


Введем вектор $\mathbf{m} \in R^{\text{n}}$ следующим образом:


\begin{equation}
\mathbf{m}_j = \dfrac1L \sum\limits_{i = 1}^{L} \textbf{y}^{i}_j.
\end{equation}


А теперь матрицу $\Sigma_t^{2}$, при построении которой, в отличие от $\Sigma_t^{1}$, процентрируем векторы $\textbf{y}^{i}$ с помощью  $\textbf{m}$: 


\begin{equation}
\Sigma_t^{2} = \dfrac{1}{L}\sum\limits_{i = 1}^{L} (\textbf{y}^{i} - \textbf{m}) (\textbf{y}^{i} - \textbf{m})^T
\end{equation}

\subsubsection{Способ № 3}

Для каждой строки $\textbf{x}^{(j)}$, $j \in {1, \ldots, n}$ матрицы $\textbf{X}$ введем величину:


\begin{equation}
\sigma_j = \sqrt{\dfrac{1}{L} \sum\limits_{i = 1}^L (\textbf{x}^{(j)}_i - \textbf{m}_i)^2}.
\end{equation}


Теперь введем матрицу $\Sigma_t^3$ следующим образом:


\begin{equation}
(\Sigma_t^{3})_{ij} = \dfrac{(\Sigma_t^{2})_{ij}}{\sigma_i \sigma_j}.
\end{equation}

\subsection{Вычислительный эксперимент}

Проведем эксперименты на одном синтетическом наборе временных рядов и двух реальных наборах временных рядов. 

1. Синтетические данные

Рассмотрим набор из $n = 4$ временных рядов представляющие собой все те же сдвинутые синусы. Длина каждого из рядов 200, а считать матрицу попарных расстояний будем по $L = 10$ значениям. Для начала, преобразуем исходные временные ряды в набор из попарных расстояний, состоящий из $\frac{n^2 + n}{2} = 10$ компонент. Проведем прогноз с помощью LSTM и ARIMA во всех трех случаях, обучая модель на первых 150 значениях временного ряда и прогнозируя следующие 50. Приведем средние ошибки прогноза:

%\newpage

\begin{comment}
\begin{figure}[h]
  \begin{minipage}{0.5\textwidth}
    \centering
    \includegraphics[width=\linewidth]{images/4_sine_corr_1_.png}
    \caption{Временные ряды попарных расстояний при 1-ом способе подсчета $\Sigma$}
  \end{minipage}\hfill
  \begin{minipage}{0.5\textwidth}
    \centering
    \includegraphics[width=\linewidth]{images/4_sine_corr_2_.png}
    \caption{Временные ряды попарных расстояний при 2-ом способе подсчета $\Sigma$}
  \end{minipage}
\end{figure}

\begin{figure}[h]
  \centering
  \begin{minipage}{0.5\textwidth}
    \centering
    \includegraphics[width=\linewidth]{images/4_sine_corr_3_.png}
    \caption{Временные ряды попарных расстояний при 3-ом способе подсчета $\Sigma$}
  \end{minipage}\hfill
\end{figure}
\end{comment}



\begin{table}[h]
\centering
\begin{tabular}{|l|l|l|l|}
\hline  
         & $\text{MAE}_{\text{LSTM}}$  & $\text{MSE}_{\text{LSTM}}$  & $\text{MAE}_{\text{ARIMA}}$\\ \hline
${\Sigma}^1$ & 0.55 & 0.5 & 1.45 \\ \hline
${\Sigma}^2$ & 0.17 & 0.04 & 0.57 \\ \hline
${\Sigma}^3$ & 0.17 & 0.04 & 0.37 \\ \hline
\end{tabular}
\caption{Ошибки прогноза на синтетическом наборе при различных способах вычисления $\Sigma_t$.}
\label{tab:stat}
\end{table}

Демонстрация прогноза для двух рядов из набора:

\newpage

\begin{figure}[h]
  \begin{minipage}{0.33\textwidth}
    \centering
    \includegraphics[width=\linewidth]{images/4_sine_corr_for_1.png}
  \end{minipage}\hfill
  \begin{minipage}{0.33\textwidth}
    \centering
    \includegraphics[width=\linewidth]{images/4_sine_corr_for_2.png}
  \end{minipage}
  \begin{minipage}{0.33\textwidth}
    \centering
    \includegraphics[width=\linewidth]{images/4_sine_corr_for_3.png}
  \end{minipage}
  \caption{LSTM прогноз попарных расстояний при соответственно 1, 2, 3-ем способе подсчета $\Sigma$.}
\end{figure}

\begin{figure}[h]
  \begin{minipage}{0.33\textwidth}
    \centering
    \includegraphics[width=\linewidth]{images/arima_sine_1.png}
  \end{minipage}\hfill
  \begin{minipage}{0.33\textwidth}
    \centering
    \includegraphics[width=\linewidth]{images/arima_sine_2.png}
  \end{minipage}
  \begin{minipage}{0.33\textwidth}
    \centering
    \includegraphics[width=\linewidth]{images/arima_sine_3.png}
  \end{minipage}\hfill
      \caption{ARIMA прогноз попарных расстояний при 1, 2, 3-ем способе подсчета $\Sigma$}
\end{figure}

\begin{comment}
    \begin{figure}[h]
  \centering
  \begin{minipage}{0.5\textwidth}
    \centering
    \includegraphics[width=\linewidth]{images/4_sine_corr_for_3.png}
    \caption{LSTM прогноз попарных расстояний при 3-ом способе подсчета $\Sigma$}
  \end{minipage}\hfill
\end{figure}
\end{comment}


2. Реальные данные 

Рассмотрим уже прогнозируемый реальный набор из $n = 24$ высоко коррелированных временных рядов. Матрицу попарных расстояний будем считать по $L = 10$ предыдущим значениям. Преобразованный набор попарных расстояний состоит из $\dfrac{24^2 - 24}{2} = 276$ временных рядов. Проведем прогноз с помощью LSTM, ARIMA поделив ряд на обучающую и тестовую выборку в отношении $7 : 3$. Получаем следующие средние ошибки прогнозов:

\begin{table}[h]
\centering
\begin{tabular}{|l|l|l|l|}
\hline  
         & $\text{MAE}_{\text{LSTM}}$  & $\text{MSE}_{\text{LSTM}}$  & $\text{MAE}_{\text{ARIMA}}$\\ \hline
${\Sigma}^1$ & 895.2 & 3265375.3 & 3874.4 \\ \hline
${\Sigma}^2$ & 152.8 & 215891.3 & 135.1 \\ \hline
${\Sigma}^3$ & 0.11 & 0.03 & 0.27 \\ \hline
\end{tabular}
\caption{Ошибки прогноза на реальном наборе при различных способах вычисления $\Sigma_t$.}
\label{tab:stat}
\end{table}



Большие ошибки MSE в первых двух случаях связаны с наличием больших скачков в исходных данных. В силу высокой корреляции исходных рядов, ряды попарных расстояний очень схожи:

\newpage

\begin{figure}[h]
  \begin{minipage}{0.33\textwidth}
    \centering
    \includegraphics[width=\linewidth]{images/real_corr_1.png}
  \end{minipage}\hfill
  \begin{minipage}{0.33\textwidth}
    \centering
    \includegraphics[width=\linewidth]{images/real_corr_2.png}
  \end{minipage}\hfill
  \begin{minipage}{0.33\textwidth}
    \centering
    \includegraphics[width=\linewidth]{images/real_corr_3.png}
  \end{minipage}\hfill
      \caption{LSTM прогноз попарных расстояний при 1, 2, 3-ем способе подсчета $\Sigma$.}
\end{figure}

\begin{figure}[h]
  \begin{minipage}{0.33\textwidth}
    \centering
    \includegraphics[width=\linewidth]{images/arima_real_1.png}
  \end{minipage}\hfill
  \begin{minipage}{0.33\textwidth}
    \centering
    \includegraphics[width=\linewidth]{images/arima_real_2.png}
  \end{minipage}\hfill
  \begin{minipage}{0.33\textwidth}
    \centering
    \includegraphics[width=\linewidth]{images/arima_real_3.png}
  \end{minipage}\hfill
      \caption{ARIMA прогноз попарных расстояний при 1, 2, 3-ем способе подсчета $\Sigma$.}
\end{figure}

\begin{comment}
    Приведем график средней ошибки прогноза попарных расстояний для каждого из рядов в 1-ом и 2-ом способе посчета:

\begin{figure}[h]
  \begin{minipage}{0.5\textwidth}
    \centering
    \includegraphics[width=\linewidth]{images/1_error_ts.png}
    \caption{Cредняя MAE прогноза попарных расстояний для каждого из рядов при 1-ом способе подсчета $\Sigma$}
  \end{minipage}\hfill
  \begin{minipage}{0.5\textwidth}
    \centering
    \includegraphics[width=\linewidth]{images/2_error_ts.png}
    \caption{Cредняя MAE прогноза попарных расстояний для каждого из рядов при 2-ом способе подсчета $\Sigma$}
  \end{minipage}
\end{figure}

Большая ошибка на временных рядах 18, 19 связана с их  высокой дисперсией. Прогнозируя попарные расстояния без временных рядов под номерами $18, 19$, получаем улучшение прогноза:

\begin{table}[h]
\centering
\begin{tabular}{|l|l|l|}
\hline
         & MAE  & MSE  \\ \hline
${\Sigma}^1$ & 785.7 & 1904480  \\ \hline
${\Sigma}^2$ & 94.1 & 38195.5 \\ \hline
${\Sigma}^3$ & 0.11 & 0.03 \\ \hline
\end{tabular}
\caption{Ошибки LSTM прогноза на реальном наборе при отказе от прогнозирования попарных расстояний рядов 18, 19.}
\label{tab:stat2}
\end{table}
\end{comment}

\subsection{Восстановление прогноза}


\bibliographystyle{unsrt}
\bibliography{biblio}

\end{document}
\bibliographystyle{unsrt}
\bibliography{biblio}

\end{document}
