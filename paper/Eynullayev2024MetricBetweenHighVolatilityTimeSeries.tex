\documentclass{article}
\usepackage{arxiv}

\usepackage[utf8]{inputenc}
\usepackage[english, russian]{babel}
\usepackage[T1]{fontenc}
\usepackage{url}
\usepackage{booktabs}
\usepackage{amsfonts}
\usepackage{nicefrac}
\usepackage{microtype}
\usepackage{lipsum}
\usepackage{graphicx}
\usepackage{natbib}
\usepackage{doi}



\title{Погружение временных рядов с высокой волатильностью в метрическое пространство}

\author{ Эйнуллаев Алтай \\
	Кафедра интеллектуальных систем\\
	Московский физико-технический институт\\
	Долгопрудный \\
	\texttt{einullaev.ae@phystech.edu} \\
	%% examples of more authors
	\And
	Яковлев Константин \\
	Кафедра интеллектуальных систем\\
	Московский физико-технический институт\\
	Долгопрудный \\
	\texttt{iakovlev.kd@phystech.edu} \\
	%% \AND
	%% Coauthor \\
	%% Affiliation \\
	%% Address \\
	%% \texttt{email} \\
	%% \And
	%% Coauthor \\
	%% Affiliation \\
	%% Address \\
	%% \texttt{email} \\
	%% \And
	%% Coauthor \\
	%% Affiliation \\
	%% Address \\
	%% \texttt{email} \\
}
\date{}

\renewcommand{\shorttitle}{\textit{arXiv} Template}

%%% Add PDF metadata to help others organize their library
%%% Once the PDF is generated, you can check the metadata with
%%% $ pdfinfo template.pdf
\hypersetup{
pdftitle={A template for the arxiv style},
pdfsubject={q-bio.NC, q-bio.QM},
pdfauthor={David S.~Hippocampus, Elias D.~Striatum},
pdfkeywords={First keyword, Second keyword, More},
}

\begin{document}
\maketitle

\begin{abstract}
	Рассматривается задача предсказания финансовых временных рядов. Основными особенностями таких временных рядов являются высокая волатильность и высокая попарная ковариация. Классическим подходом к решению задачи является выполнение прогноза в исходном пространстве. Новый метод заключается в переходе в пространство попарных расстояний между временными рядами, осуществлении прогноза в нем и переходе обратно в исходное пространство. Для его реализации необходимо ввести функцию расстояния между временными рядами (метрику), которая должна удовлетворять определенным свойствам. В данной статье изучаются  эти свойства и проводятся сравнения различных метрик на основе численных экспериментов.

\end{abstract}


\keywords{Временные ряды \and Метрика \and Ковариация}

\section{Introduction}


\end{document}
