\documentclass{article}
\usepackage{arxiv}

\usepackage[utf8]{inputenc}
\usepackage[english, russian]{babel}
\usepackage[T1]{fontenc}
\usepackage{url}
\usepackage{booktabs}
\usepackage{amsfonts}
\usepackage{nicefrac}
\usepackage{microtype}
\usepackage{lipsum}
\usepackage{graphicx}
\usepackage{natbib}
\usepackage{doi}



\title{Погружение временных рядов с высокой волатильностью в метрическое пространство}

\author{ Эйнуллаев Алтай \\
	Кафедра интеллектуальных систем\\
	Московский физико-технический институт\\
	Долгопрудный \\
	\texttt{einullaev.ae@phystech.edu} \\
	%% examples of more authors
	\And
	Яковлев Константин \\
	Кафедра интеллектуальных систем\\
	Московский физико-технический институт\\
	Долгопрудный \\
	\texttt{iakovlev.kd@phystech.edu} \\
	%% \AND
	%% Coauthor \\
	%% Affiliation \\
	%% Address \\
	%% \texttt{email} \\
	%% \And
	%% Coauthor \\
	%% Affiliation \\
	%% Address \\
	%% \texttt{email} \\
	%% \And
	%% Coauthor \\
	%% Affiliation \\
	%% Address \\
	%% \texttt{email} \\
}
\date{}

\renewcommand{\shorttitle}{\textit{arXiv} Template}

%%% Add PDF metadata to help others organize their library
%%% Once the PDF is generated, you can check the metadata with
%%% $ pdfinfo template.pdf
\hypersetup{
pdftitle={A template for the arxiv style},
pdfsubject={q-bio.NC, q-bio.QM},
pdfauthor={David S.~Hippocampus, Elias D.~Striatum},
pdfkeywords={First keyword, Second keyword, More},
}

\begin{document}
\maketitle

\begin{abstract}
	Рассматривается задача прогнозирования финансовых временных рядов. Основными особенностями таких временных рядов являются высокая волатильность и высокая попарная ковариация. Классическим подходом к решению задачи является выполнение прогноза в исходном пространстве. Новый метод заключается в переходе в пространство попарных расстояний между временными рядами, осуществлении прогноза в нем и переходе обратно в исходное пространство. Для его реализации необходимо ввести функцию расстояния между временными рядами (метрику), которая должна удовлетворять определенным свойствам. В данной статье изучаются  эти свойства и проводятся сравнения различных метрик на основе численных экспериментов.

\end{abstract}


\keywords{Временные ряды \and Метрика \and Ковариация}

\section{Introduction}

В текущей статье исследуется задача погружения временных рядов в метрическое пространство. Таким образом, набору временных рядов ставится в соответствие матрица попарных расстояний и появляется возможность перейти от прогнозирования набора временных рядов к прогнозированию матрицы попарных расстояний. При этом выбор метрики осуществляется так, чтобы по полученной матрице расстояний можно было восстановить прогноз для набора временных рядов.

В статистике, обработке сигналов и многих других областях под временным рядом понимаются последовательно измеренные через некоторые (зачастую равные) промежутки времени данные. Прогнозирование временных рядов заключается в построении модели для предсказания будущих событий основываясь на известных событиях прошлого, предсказания будущих данных до того как они будут измерены. Типичный пример — предсказание цены открытия биржи основываясь на предыдущей её деятельности.

Одними из хорошо известных, классических методов прогнозирования временных рядов являются экспоненциальное сглаживание (англ. Exponential Smoothing) \cite{ES}, LSTM (англ. Long Short-Term Memory) \cite{LSTM}, ARIMA  (англ. autoregressive integrated moving average) \cite{ARIMA}. Главным отличием исследуемого метода от вышеперечисленных является то, что временные ряды прогнозируются при помощи прогнозирования матрицы попарных расстояний.

В качестве простейшей метрики рассматривается ковариация между временными рядами. \cite{Boyd} Таким образом, для набора временных рядов получаем матрицу ковариации. Стоит заметить, что матрица ковариации (матрица попарных расстояний) вычисляется в каждый момент времени. Альтернативные варианты метрики выбираются из класса ядер \cite{shawe2004kernel}.

Численные эксперименты проводятся на трех видах данных: синтетические, сигналы коры головного мозга, финансовые временные ряды. Эксперимент состоит из выполнения прогноза временного ряда при помощи прогнозирования матрицы попарных расстояний. В качестве прогностической модели выбирается линейная регрессия с использованием SSA (англ. Singular Spectrum Analysis) \cite{vautard1992singular}. По результатам экспериментов проводится анализ точности прогноза и его устойчивости в зависимости от выбранной метрики и вида данных. Цель эксперимента состоит в оптимальном выборе функции попарных расстояний для выполнения прогноза.

\bibliographystyle{unsrt}
\bibliography{biblio}

\end{document}
